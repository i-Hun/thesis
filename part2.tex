\chapter{Практическая часть} \label{chapt2}
\section{Примеры использования методов text mining в социологии} \label{}
В 2011 году с помощью методов text mining были выявлени интересные различия в освещении деятельности политических партии и политиков бельгийскими интернет-СМИ \cite{MediaCoverage2012}. Данное исследование позволило выявить политические предпочтения различных новостных изданий на основе определения тональности статей на их сайтах о различных политических субъектах. В исследовании использовались методы сбора и анализа тональности текста (sentiment analysis) из библиотеки \href{http://www.clips.ua.ac.be/pages/pattern}{Pattern} (язык программирования --- Python).

Похожие методы были применены для выявления разиличий в освещении событий, приведших к восстанию 2011 года в Египте, египетискими государственными и негосударственными СМИ. Было показано, что правительственные СМИ при освещении таких событий акцентировали внимание на угрозе дестабилизации и терроризма и старались рассказывать проведении реформ в стране. Независимые же СМИ наоборот были нацелены на мобилизацию в целях противостоянию режиму и фактически игнорировали действия правительства. Основной метод исследования --- латентное размещение Дирихле (LDA).



\clearpage