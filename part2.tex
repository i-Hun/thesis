\chapter{Практическая часть. Исследование образа губернатора омской области в местных Интернет-СМИ} \label{chapt2}
\section{Определение целей исследования} \label{sect2_1}
В данной части работы мы разработаем и проведём исследование, на примере которого будут показаны возможности метода интеллектуального анализа текста в социологии. Цель исследования на данном этапе состоит в том, чтобы с помощью интеллектуального анализа текста выявить некоторые характеристики дискурса о мэре г. Омска в местных Интернет-СМИ. Такими характеристиками являются:
\begin{enumerate}
\item Распределение статей с упоминанием мэра во времени. Нас будет интересовать, в какие месяцы или дни недели активизируется соответствующий дискурс. В контексте этой задачи мы сравним распределение статей во времени из генеральной совокупности и распределение статей из выборочной совокупности с тем, чтобы определить, значимо ли они различаются. В зависимости от полученных результатов можно выдвинуть гипотезы, объясняющие полученное распределение.
\item Количество комментариев. Для определения заинтересованности читателей в данной теме, сравним количество комментариев к статьям о мэре со средним количеством комментариев.
\item Тема документа, в котором упомянут мэр. С помощью алгоритмов тематического моделирования мы определим тематический контекст статей из выборки, изучим его распределение во времени и сравним его темами в генеральной совокупности на предмет наличия общих и особенных тем. Попутно мы оценим эффективность работы алгоритма тематического моделирования. % надо сравнить темы, которые алогоритм присвоил статьям о мэре, когда они были частью генеральной совокупнстьи. если алгоритм выделил эти статьи в собоую тему, то всё норм. А ещё надо посмотреть, в какием темы раскидало статьи о мэре, при выделение тем в генеральной совокупности. 
\item Попытка анализа настроений. Нам будет произведено сравнение эмоций по выборке, где упомянут мэр и остальной, а так же в различных темах.
% полноценный анализ тональности не будет использован так как нет готовых инструментов и он плохо работает на больших документах.
% Возможные варианты: 1) Выделить наиболее употребительные слова как тут: http://globalvoicesonline.org/2014/03/02/sentiment-analysis-of-russian-tweets-about-war-in-crimea/
% 2) Выделить темы, где эмоции проявляются наиболее сильно. Таким образом можно определить самые важные темы.
% 3) Тональный анализ можно произвести на названиях статей. Для анализа использовать sentimental или sentistrength
\end{enumerate}
\section{Оценка доступности и характера данных. Сбор данных.} \label{sect2_2}
Прежде чем начать сбор данных, нам придётся поставить перед собой несколько вопросов, представляющих особенную трудность в данного типа исследованиях. А именно, необходимо определить, что будет являться носителем знаний по исследуемой проблеме (т. е. эмпирическим объектом исследования), каковы границы генеральной совокупности, какой метод будет являться адекватным для построения выборочной совокупности, как определить качественные и количественные характеристики выборки, каковы критерии репрезентативности выборки \cite{methodlogy_internet}.

\textbf{Определение эмпирического объекта.} В самом общем виде можно сказать, что источником знаний о проблемах, затронутых в данном исследовании является статьи в Интернет-СМИ Омской области. Углубляясь дальше, мы должны решить какие аспекты статьи нас интересуют. Статья в Интернет-СМИ -- не просто один лишь неструктурированный текст. Это документ, который имеет свою структуру. В этой структуре нас буду интересовать такие элементы как собственно текст, название, дата публикации, количество комментариев, сами комментарии, принадлежность к тому или иному СМИ. Первая причина, по которой они были выбраны состоит в представленности этих элементов в статьях каждого из рассматриваемых нами Интернет-СМИ. Количество просмотров и ключевые слова (тэги), например, на некоторых ресурсах бывают не указаны. Другая причина -- достаточность данных элементов для решения исследовательских задач.

\textbf{Определение генеральной совокупности.} Генеральную совокупность в данном исследовании составляют статьи Интернет-СМИ г. Омска. Интернет-СМИ — веб-сайт, ставящий своей задачей выполнять функцию средства массовой информации (СМИ) в сети Интернет. По данным Агентства Региональных Исследований за июнь 2014 года в Омске работает около 18 Интернет-СМИ с месячным количеством уникальных посетителе в месяц более 10000\cite{ari_rating}. 

\textbf{Определение выборочной совокупности.} Использование данных со всех возможных ресурсов -- очень трудоёмкая задача, поскольку требует практически полного переписывания соответствующей части программы, ответственной за собственно сбор данных, и частичной переработки модуля предварительной обработки. Вполне привычным для социолога решением будет конструирование выборки. Однако как рассчитать выборку, если не известны объёмы генеральной совокупности. А даже если мы знали количество статей каждого из рассматриваемых Интернет-СМИ за любой промежуток времени, разве было бы корректно использовать традиционные методы определения выборочной совокупности в такого типа исследованиях? Эта аналогия видится некорректной по причине кардинального различия эмпирических объектов -- человека и текста. При определении людей в качестве эмпирических объектов исследования социолог как правило предполагает, что они в равной степени могут служить источником информации о проблеме. Исключение из этого правила встречается, когда исследователь отдельно изучает мнение экспертов. Но экспертные опросы -- это отдельная часть исследования, в которой как правило используются другие методы сбора и анализа информации.

В нашем случае, исходя из цели исследования -- определение образа мэра, транслируемого Интернет-СМИ в обществе, -- важна не сама статья, а влияние, оказываемое ей на общество. Именно это влияние и определяет степень, с которой статья может служить источником информации о проблеме. Его прямым индикатором служит количество просмотров данной статьи. Но не все Интернет-СМИ предоставляют эту информацию, поэтому можно опереться на косвенный индикатор -- количество просмотров всех статей исследуемого ресурса. То есть мы предполагаем, что чем больше совокупное количество просмотров у одного ресурса, тем более сильно влияние каждой его отдельной статьи. Для верности этого тезиса необходимо только чтобы общее количество статей в каждом из ресурсов не сильно отличалось друг от друга. {\underline{привести данные}}

Приведённые выше рассуждения позволяют считать, что при выборе Интернет-СМИ, статьи которых будут подвергнуты анализу, стоит опираться на общее количество просмотров. Однако это не даёт ответа на вопрос, сколько и какие статьи должны быть отобраны. Способов расчёта этих значений на сегодняшний день нет. Существуют публикации отдельных исследователей из разных отраслей, разрабатывающих свой методологический аспект при исследований текстов в сети Интернет. Определение выборочной совокупности прежде всего зависит от того, что информация о чём важно для исследования. На основании этого определяет эмпирический объект (это может быть текст, комментарий, отдельное высказывание и др.) и принцип определения его важности (например, степень влияние на общество). В любом случае сохраняет один принцип -- из выборки необходимо получить репрезентативное подмножество, -- но исследователи пытаются достичь его разными путями.

З. Папачарисси в при исследовании блогосферы определяла в качестве генеральной совокупности все блоги, расположенные на платформе blogger.com. Она объясняет свой выбор тем, что это наиболее популярный и большой по числу блоггеров англоязычный сайт, который предоставляет возможности для персональных публикаций в стиле любительской журналистики. Любой блог, по мнению Папачарисси, размещённый это этом сайте, представлял собой единицу анализа, отвечающую по своим характеристикам признакам принадлежности к генеральной совокупности. Однако нельзя согласиться, что блоги с blogger.com репрезентативны относительно всей блогосферы. Выборочную совокупность блогов Папачарисии составляла использую случайную отправную точку и случайный выборочный интервал. Однако исследователем не было оговорено, какием именно данные вводились в поисковую систему для поиска релевантных блогов и какие именно блоги считались релевантными, сколько блогов входило в генеральную совокупность и почему было отобрано именно 260. К тому же использование поисковых систем для поиска блогов выглядит сомнительно: алгоритмы данных систем неизвестны исследователю и нельзя сказать, почему были отобраны эти блоги, а не иные.

Этот пример исследования был приведёт нами, чтобы проиллюстрировать отсутствие единой позиции в способах определения выборочной и генеральной совокупности в Интернет-исследованиях. Каждый исследователь придумывает сам, каким способом наиболее полно реализовать принципы выборки.

В нашем случае необходимо определить несколько Интернет-СМИ, все статьи которых будут отобраны для исследования. Мы выяснили, что при определении значимости, веса статьи определяющей характеристикой является количество просмотров. Хотя Интернет-СМИ в Омске и немало, не все из них одинаково крупны. Судя по тем же данным АРИ, в Омск существует всего четыре новостных ресурса, страницы которых просматривают более одного миллиона раз в месяц. В процентном отношении они занимают 65\% рынка омских Интернет-СМИ. Представляется, что анализ статей, получивших более половины всех просмотров является достаточным основанием для выделения их в качестве выборочной совокупности, по результатом анализа которой можно будет делать выводы об омских Интернет-СМИ в целом. Таким образом в исследовании будут проанализированы все новостные статьи с сайтов \href{http://gorod55.ru/}{gorod55.ru}, \href{http://bk55.ru/}{bk55.ru}, \href{http://ngs55.ru/}{ngs55.ru}, \href{http://omskinform.ru/}{omskinform.ru} за период с 1 сентября 2013 по 1 сентября 2014. Новостными статьями будут считаться те, которые публикуются на данном ресурсе в разделе <<Новости>>. Статьи из категорий <<Работа>>, <<Объявления>>, <<Блоги>> и др. в анализе не участвуют.

Определившись с данными, которые необходимо собрать, нужно решить, каким способом это сделать, т. е. с использованием каких инструментов и технологий будет производится сбор данных. Для этого мы будем использовать язык программирования Python. Основанием для такого выбора является его простота, поддержка многопоточности, что полезно для более быстрого сбора данных, наличие сторонних библиотек, что позволяет избежать написание рутинного кода, а также то, что обработка и анализ данных также будет производится на этом языке -- это обеспечивает некоторую консистентность исследования. Ближайшей альтернативой данному решению видится использование программной платфоры node.js из-за хорошей поддержки асинхронных запросов (и, следовательно, высокой скорости) и наличия множества качественных библиотек для сбора данных или языка R, который традиционно популярен в академической среде для сбора и анализа данных.

Для хранения данных будет использована база данных MongoDB. Как говорилось выше, в БД будут присутствовать следующие поля: называние статьи (title), содержимое статьи (content), ссылка на статью (url), дата публикации (date), количество комментариев (commentsCount) и список комментариев к статье (comments). Статьи с каждого источника будут храниться в отдельной коллекции.

Результаты сбора данных следующие:
\begin{itemize}
\item C сайта gorod55.ru было собрано 6302 статьи
\item Больше всего новостных статей за указанный промежуток времени было опубликовано на bk55.ru -- 14078 статей  на bk55.ru
\item Наименьшее количество статей -- 4780 -- было найдено на ngs55.ru
\item 8727 статей по указанным параметрам было собрано с сайта omskinform.ru
\end{itemize}

Всего таким образом в анализе участвовало 33887 статей.

На этом этапе крайне важно контролировать корректность и полноту собираемых данных. Сложнее всего было с сайтом bk55.ru, поскольку в нём использовались несколько различных шаблонов для отображения информации, каждый из которых необходимо было отследить и создать под него набор правил для извлечения данных.

\section{Предварительная обработка данных} \label{sect2_3}

Предварительная обработка данных -- один из важнейших этапов в анализе текста. Наша цель на этом этапе -- удаление несущественных и вносящих помехи данных и преобразование данных к удобному для анализа виду.

На самом деле удалять несущественные данные мы начали ещё на этапе сбора данных, поскольку перед записью в базу данных весь текст, если это было необходимо, очищался от html-разметки. Преобразование же данных на том этапе заключалось в конвертации текста, содержащего информацию о дате публикации, в специальный тип данных, позволяющий обращаться к этим данным как к дате, например, производить выборку статей за определённый период.

Следующий шаг в предварительной обработке данных заключается в удалении из каждой статьи признаков, свидетельствующих о её принадлежности к какому-либо источнику. Если посмотреть на полученные тексты, то можно увидеть, что редакция каждого СМИ устанавливает собственные правила оформления документов, касающиеся оформления ссылок на источники данных, фотографий, указание имён авторов. В случаем если эти отличительные черты не будут устранены, алгоритмы тематического моделирования, которые мы в дальнейшем собираемся применить к собранному корпусу текстов, будут стремиться образовать темы вокруг источников. Процедура унификации статей из различный источников достаточно трудоёмка и требует ручного анализа множества статей с каждого из них, с тем чтобы выявить в них специфические черты для каждого сайта. Такими чертам могут быть имена журналистов данного издания или правила оформления фото и видео материалов (например, около каждой фотографии может указываться копирайт).

Например, чтобы удалить имена журналистов из текстов статей на сайте bk55.ru, необходимо было во-первых, составить их список. Для составления списка, была написана небольшая программа, выводящая два последних слова каждого документа, если они начинались с заглавной буквы (как правило имена авторов указывались в конце документа, хоть и не всегда). Из полученного списка примерно в пятьсот пар были вручную отсеяны пары, не являющиеся именем и фамилией. Те пары из этого списка, которые встречались больше двух раз, считались нами именем и фамилией журналистов сайта bk55.ru. На последнем этапе фамилии журналистов удалялись из каждого документа. К тому же, так как после имён журналистов часто указывалась другая мета-информация (главным образом ссылки источники информации), то также удалялся весь текст после имён, если по размеру этот текст не превышал определённое количество символов (чтобы предотвратить удаление не мета-информации). 

После устранения специфической информации данные из различных источников объединялись в единый корпус и подвергались дальнейшей обработке. Обработка заключалось в следующем:
\begin{enumerate}
\item Перевод текста в нижний регистр
\item Токенизация % неразрывные пробелы
\item Удаление пунктуации
\item Стемминг
\item Удаление стоп-слов
\item Замена слов
\end{enumerate}

Что касается перевода текста в нижний регистр и удаления пунктуации, то это достаточно тривиальные процедуры, не требующие особых объяснений.

Другим этапов предварительной обработки текста является токенизация. Именно с неё начинается обработка естественного языка как наука и как конкретная деятельность \cite{Webster1992}. Под токенизацией понимают процесс сегментации текста на отдельные части, называемые токенами. Именно токены являются теми первичными элементами, которые непосредственно участвуют в процессе анализа. 

Выделяют два основных признака токена -- лингвистическая значимость и методологическая полезность \cite[стр. 1106]{Webster1992}. В языках с иероглифической письменностью токенизация является серьёзной проблемой, поскольку один иероглиф может обозначать как морфемы (в таком случае он не удовлетворяет требованиям для того, чтобы считаться токеном), так и целые слова. В английском и русском языках проблема токенизации не стоит так остро и чаще всего токены определяются через пробелы между словами и знаки препинания. Тем не менее, даже в этих языках существуют определённые нюансы.

Нами было протестировано несколько алгоритмов токенизации (токенайзеры TreebankWordTokenizer, WordPunctTokenizer, PunctWordTokenizer и WhitespaceTokenizer из программы NLTK и токенайзер из Pattern). Корректнее всех выделял токены токенайзер из программы \href{http://www.clips.ua.ac.be/pattern}{Pattern}. Например, он единственный интерпретировал url'ы как цельные токены, не выделяя в них отдельные сегменты, на основе знаков препинания.

После токенизации и удаления токенов, являющихся знаками препинания, мы перешли от представления документов как набора символов к документам как списку слов. Дальнейшие наши шаги будут направлены на уменьшение длинны этого списка, т. е. на снижение как общего количества токенов, так и количества их уникальных единиц. Необходимость этих шагов обусловлена желанием снизить вычислительную сложность при анализе данных.

Первый шаг направлен на снижение количества уникальных токенов. Для компьютера различные формы одного и того же слова являются совершенно разными словами. Существует два способа для приведения словоформ к одной лексеме. Первый, самый простой, называется стемминг. Он состоит в отсечении слово- и формообразующих частей -- префиксов, суффиксов, окончаний, в результате чего остаётся основа слова -- неизменная часть, выражающая его лексическое значение.

Более сложным подходом к решению проблемы унификации словоформ является лемматизация. Лемматизация -- это процесс приведения словоформы к лемме — её нормальной (словарной) форме. В русском языке нормальная форма имени существительного имеет именительный падеж и единственно число, для прилагательных добавляется требование мужского рода, а глаголы, деепричастия и причастия в нормальной форме должны стоят в инфинитиве.

Для постановки слова в нормальную форму необходимо иметь словарь, где для каждого слова определены его характеристики, т. е. часть речи, падеж, число, род, форма глагола (если это глагол). Создание такого словаря требует колоссальных трудов. В отличии от этого, стемминг предполагает наличие лишь списка приставок, суффиксов и окончаний, количество которых исчисляется несколькими десятками. К счастью, для русского языка существует так необходимый для лемматизации словарь, созданный в рамках проекта \href{http://opencorpora.org/}{OpenCorpora}. Использующая этот словарь программа \href{https://pymorphy2.readthedocs.org/}{pymorphy2} позволяет приводить слова к нормальной форме.

Между вышеозначенными способами мы выбрали лемматизацию, поскольку получаемые в результате этого процесса леммы легче интерпретировать, чем усечённые основы слов, значение которых не всегда легко восстановить.

Дальнейшие усилия по уменьшению количества токенов связаны с удалением так называемых стоп-слов. Эти слова сами по себе почти не неся полезного смысла, тем не менее, необходимы для нормального восприятия текста. Чаще всего к разряду стоп-слов относятся служебные части речи -- предлоги, союзы, частицы. Будучи широко распространёнными в тексте, они мало могут сказать о его теме.

В качестве базы для списка стоп-слов был использован список русских стоп-слов из программы NLTK. Однако его нельзя считать достаточно полным. Включая в себя 151 слово данный список покрывает лишь самые основные случаи. Для его пополнения необходимо обратиться к собранным ранее данным. На их основе был составлен список наиболее часто встречающихся в корпусе токенов. Среди них были выбраны несколько десятков слов, наиболее точно подходящие под описание стоп-слов (это, который, такой, некоторый, другой, тот и др.), которые затем были добавлены в соответствующий список. Представляется, что такой список, дополненный словами, выбранными из числа наиболее распространённых, является достаточно полным, поскольку стоп-слова по своему характеру всегда относятся к наиболее часто встречающимся в тексте. Редкие слова как правило свидетельствуют о принадлежности текста к какой-либо теме, а потому не могут относиться к разряду стоп-слов.

В заключение, для удобства анализа была произведена замена некоторых слов. Данная замена включала в себя во-первых, раскрытие аббревиатур (рф $\to$ россия, ул $\to$ улица и др.), а во-вторых, лемматизацию токенов, которые не были лемматизированы автоматически (расина $\to$ расин, парка $\to$ парк). Данный шаг не является обязательным и может быть без последствий пропущен.

Как видно, в общих чертах данный набор процедур повторяет составляющие предварительной обработки данных из методологии CRISP-DM.

Необходимо отметить, что после каждой операции с данными на этапе предварительной обработки следует контролировать последствия производимых изменений. Такой контроль поможет выявить проблемы на раннем этапе, что убережёт от лишней работы в будущем \footnote{Например, одной из таких проблем, выявленных на раннем этапе, было наличие в текстах некоторых СМИ неразрывных пробелов. Они мешали токенизации, поскольку сегментация производилась по обычным пробелам. Решением стала замена всех неразрывных пробелов на обычные.}. Легче всего производить контроль через анализ изменений в списке наиболее часто встречающихся слов.

\section{Анализ данных}

\subsection{Тематическое моделирование}
% обзор методов
% история
% задачи
% LDA
% применение LDA
Одна из главных задач данного исследования -- выявление тем собранных ранее статей. Данная задача известна как тематическое моделирование (topic modeling).

Тематическое моделирование активно развивается последние двенадцать лет и находит своё применение в широком спектре приложений. Оно применяется для выявления трендов в научных публикациях, для классификации и кластеризации документов, изображений и видеопотоков, для информационного поиска, в том числе многоязычного, для тегирования веб-страниц, для обнаружения текстового спама, для рекомендательных систем и других приложений \cite[стр. 4]{voroncov2013}. 

Тематическое моделирование постепенно находит признание и среди социологов. % рассказать о примерах использования LDA в соц. исследованиях не в России.

В российской социологии подобного вида исследования проводились исследовательски коллективом Лаборатории интернет-исследований Санкт-Петербургского филиала ВШЭ \cite{kolcovalda}. Материалом для тематического моделирования послужили записи 2000 самых популярных блогеров по рейтингу популярности Живого Журнала. Для тематического моделирования в данном исследования была использована созданная в лаборатории программа TopicMiner, которая сменила использовавшийся ранее Stanford Topic Modeling Toolbox. Обе этих программы реализовывают алгоритм латентного размещения Дирихле с сэмплированием Гиббса.

Построение тематическое модели может рассматриваться как задача одновременной кластеризации документов и слов по одному и тому же множеству кластеров, называемых темами. В терминах кластерного анализа тема -- это результат би-кластеризации, то есть одновременно кластеризации и слов и документов по их семантической близости. Обычно выполняется нечёткая кластеризация, то есть документ может принадлежать нескольким темам в различной степени. Таким образом, сжатое семантическое описание слова или документа представляет собой вероятностное распределение на множестве тем. Процесс нахождения этих распределений и называется тематическим моделированием \cite{korshunov2012}.

Что касается конкретных методов тематического моделирования, то одним из первых был предложен вероятностный латентный семантический анализ (probabilistic latent semantic analysis, PLSA), основанный на принципе максимума правдподобия, как альтернатива классическим методам кластеризации, основанным на вычислении функций расстояния. Вслед за PLSA в 2003 году был предложен метод латентного размещения Дирихле (latent Dirichlet allocation, LDA) \cite{LDAOrigin} и его многочисленные обобщения \cite{NeedlesInAHaystack}, \cite{LDASurvey}. В том числе благодаря этим обобщениям LDA безусловно лидирует среди вероятностных тематических моделей.

Эти обобщения учитывают специфические переменные, что улучшает работу алгоритма в приложении к конкретным задачам. Например, когда исследуемые документы имеют дату публикации, можно применить модель Topics over Time LDA, которая более корректно показывает изменение присутствия тем во времени \cite{ToTLDA}. Другие модификации могут учитывать такую переменную как авторство текста, ведь тексты одного автора имеют большую вероятность относиться к определённому набору тем \cite{authorLDA}.

Параллельно множеству обобщений, существует две основных разновидности методов LDA, отличающихся методами оценивания, т. е. нахождения значения параметров модели, при которых наблюдаемая обучающая выборка максимально правдоподобна \cite{kolcovaJJ}, \cite[стр. 1]{HoffmanBB10}. Первая разновидность -- вариационная модель LDA, чья численная схема основана на принципе максимизации функции правдоподобия. В рамках данной модели реализовано предположение о том, что дна функция Дирихле описывает лишь одно распределение (одного слова по темам или одного документа по темам); соответственно поиск распределение каждого слова и каждого документа по темам приводит к работе с огромными матрицами. Таким образом размерность матриц существенно зависит от размера словаря, поэтому качественный препроцессинг документов играет важную роль в тематическом моделировании. Кроме того, наличие произведение большого числа функции приводит к множеству локальных максимумов в функции правдоподобия. Таким образом, метод максимального правдоподобия может приводит не к оптимальным результатам, так как этот метод лишь даёт гарантия попадания в один из локальных максимумов, но не позволяет находить наибольший максимум среди множества локальных экстремальных точек.

Второй разновидностью метода LDA является метод сэмплирования Гиббса -- статистический алгоритм на основе методов Монте-Карло, в котором строится марковская цепь, сходящаяся в апостериорному распределению тем, по которым далее строятся оценки параметров. Сэмплирование Гиббса позволяет эффективно находить скрытые темы в больших корпусах текстов. Сложно сказать, какой из двух подходов лучше. Многое зависит от особенностей конкретной реализации.

В данном исследовании используется подход, разработанный Мэтью Хоффманом \cite{HoffmanBB10} и реализованный в программе Gensim. Он относится к первой группе алгоритмов -- вариационной модели LDA. Данный выбор обусловлен тем, что в рамках выбранных инструментов эта программа является самым популярной и хорошо документированным вариантом.

% Общее количество токенов 118718. Найдёно 49271 редких токенов. Ошибки, цифры, ссылки, английские слова (в том числе написанные транслитом), имена собственные, просто редкие

\subsection{Подготовка данных}
Общее количество токенов 118718. Найдено 49271 редких токенов (1 раз в корпусе)

\subsection{Определение оптимального количества тем и их идентификация}
% В Exploiting affinities between topic modeling and the sociological perspective on culture: Application to newspaper coverage of U.S. government arts funding было выделено 12 тем.
Определение оптимального числа тем -- важная подзадача в тематическом моделировании, поскольку её решение существенно влияет на осмысленность получаемого набора тем. Занижение числа тем приводит к чрезмерно общим результатам. Завышение приводит к невозможности разумной интерпретации. Оптимальное число тем зависит от числа документов в анализируемом корпусе: в малых корпусах оптимальным является, как правило меньшее число тем. Согласно оригинальному исследованию \cite{LDAOrigin}, оптимальное число тем для корпуса из 16333 новостных статей составило 100, тогда как для корпуса из 5225 аннотаций научных статей -- 50. Однако не существует однозначного метода определения оптимального количества тем, и часто это количество определяется "на глазок", исходя из личного мнения исследователя. 

В данном исследовании использовался метод определения оптимального количества тем на основе перплексии -- это стандартный способ оценки качества модели. Перплексия равняется экспоненте от минус усреднённого логарифма правдоподобия и показывает, насколько хорошо модель приближает наблюдаемые частоты появления слов в документах. Качество модели тем выше, чем меньше перплексия.

Для измерения перплексии необходимо разделить выборку на две части -- тренировочную -- которая будет использоваться при построении модели, и текстовую, на которой будет проверяться точность предсказаний модели. В данном исследовании контрольную выборку составляли 10\% случайно выбранных документов, остальные использовались для тренировки модели. Модели рассчитаны для количества тем от 5 до 100 с шагом в 5.

Используя стандартные методы расчёта перплексии из программы Gensim мы получили следующие результаты:

\begin{figure}
	\centering
    \includegraphics[totalheight=10cm]{../code/output/perplexity_exp2.png}
    \caption{Изменение перплексии в зависимости от количества тем в программе Gensim}
    \label{fig:perplexity_gensim}
\end{figure}

Как видно из графика, в нашем случаем по мере увеличения количества тем перплексия также увеличивается, в то время как должен происходить обратный процесс -- большее количество тем лучше описывает распределение. Скорее всего это недостатки реализации расчёта перплексии в Gensim, поскольку сам автор программы признаёт наличие проблемы у некоторых пользователей\footnote{https://groups.google.com/d/msg/gensim/TpuYRxhyIOc/JbTjqCcC6uYJ}.

Попробуем рассчитать перплексию с помощью другого инструмента и используем для этого популярную программу для тематического моделирования Mallet. Как упоминалось ранее, данная программа использует совершенно другой подход к тематическому моделированию, поэтому графики, полученные в ней, сильно отличаются от предыдущих:

\begin{figure}
	\centering
    \includegraphics[totalheight=10cm]{../code/output/perplexity_mallet.png}
    \caption{Изменение перплексии в зависимости от количества тем в программе Mallet}
    \label{fig:perplexity_mallet}
\end{figure}

Как видно из графика, мы получили несколько локальных минимумов перплексии при 45, 60 и 85 темах.

Какие могут быль альтернативы расчёту перплексии? Во-первых, можно использовать алгоритмы тематического моделирования, которые автоматически подбирают оптимальное количество тем. Таким алгоритмом является, например, иерархический процесс Дирихле (hierarchical Dirichlet process, HDP), который напоминает LDA с то разницей, что данный подход относится к непараметрическим, а модель сама определяет оптимальное количество тем. Так как в Gensim присутствует реализация данного алгоритма, не составит труда применить его на нашей выборке.

В результате иерархического процесса Дирихле мы получили более 150-и тем. Однако данное количество тем довольно сложно  интерпретировать, ведь нам необходимо проанализировать каждую тему и дать ей название.

Ещё один опробованный нами способ решения данной задачи описан в статье под названием <<О нахождении естественного числа тем в LDA: некоторые наблюдения>> \cite{Arun_KL}. В ней автор предлагает использовать  расстояние Кульбака — Лейблера как способ оценки качества модели. Чем меньше указанное расстояние, тем лучше модель. Наш расчёт этого расстояния на моделях с разным количеством тем, построенных в Gensim, дал следующий график:

\begin{figure}
	\centering
    \includegraphics[totalheight=10cm]{../code/output/kl_div.png}
    \caption{Изменение расстояния Кульбака — Лейблера в зависимости от количества тем в программе}
    \label{fig:kl_div}
\end{figure}

Из него видно, что оптимальное количество тем равняется 15, 25, 50.

В конечном итоге после сравнения моделей с разным количеством тем, мы выбрали модель с 50-ю темами, поскольку сгенерированные ей темы легче всего подвергались интерпретации и сильнее всего отличались друг от друга.

Темы получились следующие:

Название темы определялись после анализа слов, которые эта тема генерирует с наибольшей вероятностью. Ниже показано как программа описывает одну из тем. Рядом к каждым словом указана вероятность, с которой оно генерируется данной темой. Как мы можем понять, эта тема имеет отношение к прогнозу погоды в области.

0.030*омск + 0.018*температура + 0.017*день + 0.015*снег + 0.014*погода + 0.014*воздух + 0.012*градус + 0.011*ветер + 0.010*область + 0.010*днём + 0.009*ожидаться + 0.009*дождь + 0.008*ночью + 0.007*выходной + 0.006*составить + 0.006*неделя + 0.006*управление + 0.006*м/с + 0.005*атмосферный + 0.005*тёплый

Не все темы можно легко идентифицировать. Несколько тем были отнесены нами в категорию <<мусорных>>.

Также мы можем рассчитать вероятностное тематическое распределение для каждого отдельного документа, выявив наиболее связанные с ним темы. Так как в LDA используется нечёткая кластеризация, каждый документ с определённой вероятностью можно отнести к любой теме. В связи с этим необходимо определить порог, который будет служить ориентиром для отнесения документа к каким-либо темам.

Здесь мы обнаруживаем, что при пороге в 0.1 один документ в среднем можно отнести к 6.64 темам, при пороге в 0.1 -- к 2.76. При пороге в 0.2 575 документов нельзя отнести ни к одной из тем.
% Для 0.1 (2.76 среднее): 3: 11250, 2: 10010, 4: 6519, 1: 4249, 5: 1652, 6: 194, 7: 3
% Для 0.01 (6.64 среднее): 6: 4988, 5: 4596, 7: 4373, 8: 3703, 4: 3655, 9: 2906, 3: 2669, 10: 1969, 2: 1396, 11: 1232, 12: 867, 13: 489, 1: 386, 14: 275, 15: 181, 16: 90, 17: 44, 18: 32, 19: 13, 20: 8, 21: 4, 23: 1

Итак, о чём пишут в омских Интернет-СМИ? На этот вопрос можно ответить по-разному. Если для каждого документа выделить одну тема, к которой он относится с наибольшей вероятностью, то получится, что самая популярная тема, которая наиболее вероятная для 1855 документов -- это школьное образование, вторая по популярности (1845 документов) -- преступления, третья (1609) -- взаимоотношения с Украиной.
% Результаты выделения одной темы для документа: 19: 1855, 43: 1845, 39: 1609, 4: 1443, 10: 1398, 1: 1347, 29: 1324, 32: 1277, 7: 1230, 16: 1098, 45: 1081, 2: 955, 11: 949, 0: 810, 6: 809, 15: 789, 41: 782, 12: 752, 25: 685, 48: 680, 36: 665, 34: 631, 3: 591, 20: 560, 5: 552, 17: 524, 31: 523, 40: 504, 26: 485, 44: 483, 37: 469, 21: 449, 9: 419, 18: 373, 38: 373, 13: 339, 28: 338, 33: 320, 30: 302, 8: 297, 22: 291, 47: 277, 23: 251, 24: 224, 42: 203, 49: 193, 27: 175, 35: 145, 46: 142, 14: 61

Но в таком случае мы упускаем важное преимущество LDA -- нечёткую кластеризацию, а именно возможность отнесения документа сразу к нескольким темам. Поэтому для выявления наиболее популярной темы разумно рассчитаем среднюю вероятность для каждой темы. [Показать результаты]
% (19, 0.047838545499559668), (43, 0.04479337210102579), (4, 0.038873601684781163), (1, 0.03832171230574407), (32, 0.037811586094280814), (39, 0.036940894019667152), (10, 0.034673390373572886), (29, 0.033336754004402343), (2, 0.031981191284096259), (7, 0.031282464451361312), (11, 0.030581387682601796), (16, 0.02864482710044694), (15, 0.025477640364014734), (0, 0.025114957973744818), (45, 0.024043480473755381), (6, 0.023684856336470952), (36, 0.023303070064770615), (25, 0.022778261693379143), (12, 0.022681096614073939), (48, 0.021490121922113314), (41, 0.020949723438316008), (20, 0.01792206620665512), (5, 0.0179139185776763), (3, 0.017616780538472199), (17, 0.016112961749655534), (26, 0.015743640609993736), (34, 0.015714254358427857), (40, 0.015254355112503349), (21, 0.014792040353948347), (31, 0.014314695985847172), (37, 0.014309689896139111), (9, 0.014022096853835162), (44, 0.012912397037953388), (38, 0.012825875525024441), (8, 0.011088072303692314), (49, 0.011017666043633191), (22, 0.01095748516356693), (13, 0.01077234704925969), (33, 0.010744741702758076), (28, 0.010600538476022341), (23, 0.010594298027741866), (30, 0.010341520605118669), (47, 0.0097213215576706091), (24, 0.0093205822034313351), (18, 0.0092205459651753217), (42, 0.0082184429960869797), (46, 0.0071585907897177732), (35, 0.0062545576507037938), (27, 0.0062389631612260899), (14, 0.0031664638410354145)

Ещё один способ решения задачи поиска наиболее популярной теме в корпусе документов -- суммирование векторов документов и поиск вероятностного распределения тем на едином векторе.
% (2, 0.057465986358771516), (39, 0.047118012203725322), (1, 0.039641327452021653), (32, 0.038186835510411542), (7, 0.037974801466161327), (43, 0.037763246861662145), (29, 0.037499097504614068), (36, 0.033569614264772306), (10, 0.033568137254111091), (16, 0.032317961704401321), (25, 0.030039406918418117), (15, 0.028708832156443487), (19, 0.028307217940765513), (4, 0.027278672385258607), (11, 0.027117243675593712), (45, 0.025442992835952229), (41, 0.021956677905551718), (6, 0.019267847080009341), (20, 0.019058866263565889), (34, 0.018927058972674788), (3, 0.01890030576657065), (0, 0.018384296018406798), (17, 0.017726507636232122), (12, 0.017713156888040798), (5, 0.017311450761041219), (37, 0.016839082048119864), (9, 0.016641041904385638), (48, 0.016079319905273008), (26, 0.014509646435122103), (22, 0.013755689225187001), (40, 0.013572614907836261), (21, 0.013178869512083793), (44, 0.013168883259968383), (31, 0.013022586809973025), (38, 0.012803245766240047), (30, 0.012011290407025482), (8, 0.011213044758897795), (13, 0.010750763818116084), (49, 0.010302974610671768), (28, 0.01029268235564381)

При таком подходе на первый план вышли темы 2 (сложно интерпретировать), 39 (Украина), 1 (региональная власть). Здесь мы встречаемся с такой проблемой, как сложность интерпретации некоторых выделенных тем. В нашем случае таких тем {написать количество}, а одна из них -- та самая тема номер 2 -- к тому же очень распространена. Проанализировав слова, которые она генерирует мы видим, что в них сложно найти что-то общее:

0.009*человек + 0.007*большой + 0.006*нужно + 0.005*город + 0.005*омск + 0.005*время + 0.005*деньги + 0.005*сделать + 0.004*хороший + 0.004*вопрос + 0.004*делать + 0.004*знать + 0.004*проблема + 0.004*журналист + 0.004*работа + 0.004*работать + 0.004*должный + 0.003*проект + 0.003*метро + 0.003*думать

К тому же вероятности, которыми тема генерирует данные слова чрезвычайно низки. Самая большая вероятность находится на уровне 0.009, в то время как в других темах примерно от 0.2 до 0.6

Одна из причин этому -- большое количество слов, которые ничего не могут сказать нам об особенностях темы. В основном это прилагательные и глаголы, которые обозначают признак предмета или его действие, но не называют сам предмет (большой, нужно, сделать, хороший и др.). Возможно, часть этих слов стоило занести в список стоп-слов.

Анализ документов, в которых проявление этой темы наиболее вероятно, также показывает сложность её интерпретации. Вот примеры заголовков некоторых из этих документов: <<Обзор блогов. Блоги – это маленькая жизнь>>, <<Сколько еще простоят хрущевки в России?>>, <<Обзор СМИ: Страшно далеки они от народа>>, <<Кустурица стоя аплодировал омским рокерам>>.

Наличие таких "мусорных" тем -- нормальное явление в тематическом моделировании, которого тем не менее надо старательно избегать, проводя качественный препроцессинг документов и выбирая оптимальное количество тем для модели.
\clearpage

\subsection{Анализ настроений}