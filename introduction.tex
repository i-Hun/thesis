\chapter*{Введение}							% Заголовок
\addcontentsline{toc}{chapter}{Введение}	% Добавляем его в оглавление

Последние несколько десятилетий наука анализа данных претерпевает существенные изменения. Появление глобальной сети Интернет и распространение персональных компьютеров привело к тому, что информации стало больше и производится она намного быстрее, чем раньше. Значительная часть человеческой коммуникации переместилась в виртуальную сферу. Практически у каждой газеты или журнала имеется электронная версия или веб-сайт, где постоянно появляются новые материалы, происходит коммуникация пользователей между собой и с редакцией, проводятся голосования и прямые трансляции. Некоторые СМИ и вовсе отказываются от бумаги и полностью перебираются в электронный формат. Предоставляя более удобные средства потребления, хранения и поиска информации, чем традиционные печатные СМИ, Интернет становится новым центром притяжения как для издателей, так и для их аудитории.

К тому же, благодаря развитию технических средств и совершенствованию алгоритмов, оперировать информацией стало проще. Обычный персональный компьютер теперь способен обрабатывать миллионы строк текста за считанные секунды.

Эти изменения открывают перед исследователями невиданные ранее перспективы. На основе наработок в области искусственного интеллекта, машинного обучения, статистики и проектировании баз данных в 80-х гг. XX века сформировалась новая междисциплинарная область знания --- data mining или интеллектуальный анализ данных. Особенность методов, объединяемых данным понятием, заключается в их способности извлекать из <<сырых>> данных ранее неизвестные нетривиальные знания. Системы data mining сейчас находятся на острие исследований и разработок в области анализа, моделирования и практического использования информации и знаний, создавая новую культуры анализа данных.

Сфера применения данных методов практически ничем не ограничена --- их можно применять везде, где имеются какие-либо данные \cite[стр. 81]{Duk2011}. Одной из таких сфер применения является интеллектуальный анализ данных --- прежде всего текста --- в социальных науках. Группа методов data mining, предназначенная для интеллектуального анализа неструктурированного текста объединяется под названием text mining -- интеллектуальная анализ текста.

В социологии анализ текстов обычно осуществляется следующими традиционными методами: дискурс-анализ, контент-анализ, когнитивное картирование и т.п. Однако, как уже говорилось, виртуальное пространство является хранилищем огромного количества текстов. В таких условиях с одной стороны возникают сложности с применением некоторых традиционных методов анализа текста, поскольку они требуют непосредственного участия исследователя в анализе каждого текста, а с другой -- появляется возможность автоматизировать процесс анализа. Здесь на помощь социальному исследователю могут прийти методы интеллектуального анализа текста. С их помощью можно получить результаты, недоступные классическим методам анализа данных -- с высокой точностью спрогнозировать результаты выборов \cite{venezuala} или предсказать популярность фильма до выхода в прокат на основе его обсуждения в сети \cite{hp_predicting}.

Однако по некоторым оценкам, многие российские социологи не знакомы с данными методами, что нельзя признать нормальным, поскольку отбрасывает отечественную социологию на 20-30 лет назад. Отсутствие соответствующей подготовки в области анализа данных приводит к поверхностному анализу эмпирических данных, в то время как важные и полезные неочевидные закономерности в данных ускользают от внимания исследователя \cite{Davydov_Knowledge}. Такое игнорирование современных методов анализа данных вполне может стать <<фатальной ошибкой>> \cite{Davidov_fatal} и привести к возникновению <<чёрной дыры>> \cite{black_hole} в российской социологии. Сказанное позволяет считать, что работа, показывающая перспективы применения методов интеллектуального анализ текстов в социологических исследованиях, является \textbf{актуальной}.

В данном исследовании мы ставим цель рассказать о таком методе анализа данных как text mining и на практическом примере показать его актуальность для социологического анализа текстов.

\textbf{Проблема исследования} заключается недостаточности наработок в области применения методов интеллектуального анализа текста в социологии.

\textbf{Объект исследования} --- методы интеллектуального анализа текста в социологическом исследовании.

\textbf{Предмет исследования} --- возможности применения интеллектуального анализа текста для задач обработки естественного языка, моделирования тем и анализа настроений в социологическом исследовании.

\clearpage