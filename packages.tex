%%% Проверка используемого TeX-движка %%%
\usepackage{ifxetex}

%%% Поля и разметка страницы %%%
\usepackage{lscape}                                 % Для включения альбомных страниц
\usepackage{geometry}                               % Для последующего задания полей

%%% Математические пакеты %%%
\usepackage{amsthm,amsfonts,amsmath,amssymb,amscd}  % Математические дополнения от AMS

%%% Кодировки и шрифты %%%
\ifxetex
  \usepackage{polyglossia}                          % Поддержка многоязычности
  \usepackage{fontspec}                             % TrueType-шрифты
\else
  \usepackage{cmap}                                 % Улучшенный поиск русских слов в полученном pdf-файле
  \usepackage[T2A]{fontenc}                         % Поддержка русских букв
  \usepackage[utf8]{inputenc}                       % Кодировка utf8
  \usepackage[english, russian]{babel}              % Языки: русский, английский
  \IfFileExists{pscyr.sty}{\usepackage{pscyr}}{}    % Красивые русские шрифты
\fi

%%% Оформление абзацев %%%
\usepackage{indentfirst}                            % Красная строка

%%% Цвета %%%
\usepackage[usenames]{color}
\usepackage{color}
\usepackage{colortbl}

%%% Таблицы %%%
\usepackage{longtable}                              % Длинные таблицы
\usepackage{multirow,makecell,array}                % Улучшенное форматирование таблиц

%%% Общее форматирование
\usepackage[singlelinecheck=off,center]{caption}    % Многострочные подписи
\usepackage{soul}                                   % Поддержка переносоустойчивых подчёркиваний и зачёркиваний
\usepackage{icomma}                                 % Запятая в десятичных дробях

%%% Библиография %%%
\usepackage{cite}                                   % Красивые ссылки на литературу

%%% Гиперссылки %%%
\usepackage[linktocpage=true,plainpages=false,pdfpagelabels=false]{hyperref}

%%% Изображения %%%
\usepackage{graphicx}                               % Подключаем пакет работы с графикой

%%% Оглавление %%%
\usepackage{tocloft}

%%% Интервалы %%%
\usepackage{setspace}

%%% Колонтитулы %%%
\usepackage{fancyhdr}

\usepackage{epigraph}
\usepackage[colorinlistoftodos]{todonotes}
%\usepackage[disable]{todonotes}  % не отображать todos
\usepackage{comment}

\usepackage{array}  % Для ширины таблиц
\newcolumntype{L}[1]{>{\centering\arraybackslash}p{#1}}
\newcolumntype{C}[1]{>{\arraybackslash}p{#1}}
\newcolumntype{R}[1]{>{\centering\arraybackslash}p{#1}}

\usepackage{multicol}
