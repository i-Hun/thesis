\chapter*{Заключение}						% Заголовок
\addcontentsline{toc}{chapter}{Заключение}	% Добавляем его в оглавление

В первой части работы показан процесс развития статистического знания, который в последние десятилетия привёл к возникновению интеллектуального анализа данных -- нового подхода к анализу данных, основанного большей частью на принципах байесовской статистики. Показаны возможности и примеры применения данного подхода в социологических исследованиях, обосновано отличие интеллектуального анализа текста от контент-анализа.

Во второй части работы продемонстрирована практическая реализация описанных в первой части принципов сбора, предварительной обработки и анализа данных, в результате чего был построен тематический профиль омских Интернет-СМИ и определены очаги социальной напряжённости по отношению к выделенным темам.

Содержательная интерпретация результатов анализа позволила выделить несколько тем, вызывающих наибольшую социальную напряжённость. Также в данной работе была выполнена дополнительная задача усовершенствования русскоязычного тонального словаря для программы SentiStength, что позволило улучшить точность определения тональности на несколько процентов по сравнению с предыдущими вариантами словарей.

В целом, можно сделать вывод, что основная цель данной работы -- теоретическое и практическое введение в довольно новый для социологов метод анализа текстовых данных -- была достигнута. Хочется надеяться, что данная работа будет полезна для специалистов нашей профессии и подтолкнёт их к пополнению своего арсенала методов изучения социальной действительности.

\clearpage